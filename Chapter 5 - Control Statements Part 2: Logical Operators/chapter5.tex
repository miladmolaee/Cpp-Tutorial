\documentclass[10pt]{beamer}
\usepackage{textpos}
\usepackage{graphicx}
\usepackage{soul}
\usepackage{xcolor,colortbl}
\usepackage{listings}
\usepackage{lipsum}
%
\definecolor{codegreen}{rgb}{0,0.6,0}
\definecolor{codegray}{rgb}{0.5,0.5,0.5}
\definecolor{codepurple}{rgb}{0.58,0,0.82}
\definecolor{backcolour}{rgb}{0.95,0.95,0.92}
\definecolor{nbackcolour}{rgb}{0.92,1,1}
\definecolor{darkblue}{rgb}{0.1,0.1,0.5}
\definecolor{outputboxcolor}{rgb}{0.2,0.3,0.5}
\definecolor{lightcyan}{rgb}{0.3,0.97,0.97}
\definecolor{dgreen}{rgb}{0,0.6,0.3}
\definecolor{lblue}{rgb}{0,0.65,1}
%
\lstdefinestyle{mystyle}{
    backgroundcolor=\color{nbackcolour},   
    commentstyle=\tiny\color{codegreen},
    keywordstyle=\color{magenta},
    numberstyle=\tiny\color{codegray},
    stringstyle=\color{codepurple},
    basicstyle=\ttfamily\scriptsize,
    breakatwhitespace=false,         
    breaklines=true,                 
    captionpos=b,                    
    keepspaces=true,                 
    numbers=left,                    
    numbersep=5pt,                  
    showspaces=false,                
    showstringspaces=false,
    showtabs=false,                  
    tabsize=2,
    xleftmargin=10pt,
    xrightmargin=6pt,
    columns=flexible
}
\lstset{style=mystyle}
%
% Change example block colors
% 1- Block title (background and text)
\setbeamercolor{block title example}{fg=white, bg=blue}
% 2- Block body (background and text)
\setbeamercolor{block body example}{bg=teal!25}
% Change alert block colors
% 1- Block title (background and text)
\setbeamercolor{block title alerted}{fg=cyan, bg=orange}
% 2- Block body (background and text)
\setbeamercolor{block body alerted}{bg=orange!25}
% Change standard block colors
% 1- Block title (background and text)
\setbeamercolor{block title}{bg=outputboxcolor, fg=white}
% 2- Block body (background)
\setbeamercolor{block body}{bg=cyan!10}
%
\usetheme{IPG}
%
\setbeamerfont{frametitle}{size=\large}
\setbeamerfont{block title}{size=\small}
\setbeamerfont{block title example}{size=\small}
\setbeamerfont{block title alerted}{size=\small}
\setbeamerfont{block body}{size=\small}
\setbeamerfont{block body example}{size=\small}
\setbeamerfont{block body alerted}{size=\small}
%
\author[miladmolaee@hotmail.com]{\large Milad Molaee}
% 
\title[C++ Programming]{C++ Programming\\\vspace{5pt}from Beginner to Expert\\\vspace{20pt}{\color{darkblue}\large Chapter 5 Control Statements: Part 2 \\ Loops and Logical Operators}}
%
\begin{document} 
%
\setbeamercolor{itemize item}{fg=red}
\frame{\titlepage}

%
\begin{frame}{Outline}
\tableofcontents
\end{frame}


\section{Introduction}
\begin{frame}{Introduction}
	\lipsum[2]
\end{frame}

\section{\textbf{\textit{\color{lblue}while}} Iteration Statement}
\begin{frame}{\textit{\color{blue}while} Iteration Statement}
	\lipsum[2]
\end{frame}

\section{\textbf{\textit{\color{lblue}do…while}} Iteration Statement}
\begin{frame}{\textit{\color{blue}do…while} Iteration Statement}
	\lipsum[2]
\end{frame}

\section{\textbf{\textit{\color{lblue}for}} Iteration Statement}
\begin{frame}{\textit{\color{blue}for} Iteration Statement}
	\lipsum[2]
\end{frame}

\section{Nested Loops}
\begin{frame}{Nested Loops}
	\lipsum[2]
\end{frame}


\section{\textbf{\textit{\color{lblue}goto}} Statement}
\begin{frame}{\textit{\color{blue}goto} Statement}
	\lipsum[2]
\end{frame}

\section{\textbf{\textit{\color{lblue}switch}} Multiple-Selection Statement}
\begin{frame}{\textit{\color{blue}switch} Multiple-Selection Statement}
	\lipsum[2]
\end{frame}


\section{\textbf{\textit{\color{lblue}break}} and \textbf{\textit{\color{lblue}continue}} Statements}
\begin{frame}{\textit{\color{blue}break} and \textit{\color{blue}continue} Statements}
	\lipsum[2]
\end{frame}


\section{Logical Operators ( {\color{lblue}\textbf{\&\&}} ), ( {\color{lblue}\textbf{||}} ), and ( {\color{lblue}\textbf{!}} )}

\begin{frame}{Logical Operators ( {\color{blue}\textbf{\&\&}} ), ( {\color{blue}\textbf{||}} ), and ( {\color{blue}\textbf{!}} )}
	\lipsum[2]
\end{frame}


\section{Confusing the Equality ( {\color{lblue}\textbf{==}} ) and Assignment ( {\color{lblue}\textbf{=}} ) Operators}
\begin{frame}{\footnotesize Confusing the Equality ( {\color{blue} == }) and Assignment ( {\color{blue} = }) Operators}
	\lipsum[2]
\end{frame}


\section{Summery and Conclusion}

\begin{frame}{Summery and Conclusion}
	\lipsum[2]
\end{frame}


\section{Exercises}

\begin{frame}{Exercises}
	\lipsum[2]
\end{frame}

\begin{frame}{\small First Program in C++: Printing a Line of Text}
	
	Consider a simple program that prints a line of text (Fig. 2.1). This program illustrates several important features of the C++ language. The text in lines 1–10 is the program’s source code. The line numbers are not part of the source code.
	
	\lstinputlisting[language=C++]{../codes/chapter-2/fig_2-1/fig_2-1.cpp}
	
	\begin{block}{\textcolor{white}{output}}
		\texttt{\small Welcome to C++!}
	\end{block}
	
\end{frame}


\begin{frame}{\small The Stream Insertion Operator and Escape Sequences}
	\centering\tiny\renewcommand{\arraystretch}{2}	
\begin{tabular}{p{0.15\linewidth} p{0.6\linewidth}}
	\rowcolor{cyan}\color{white} Escape sequence & \color{white} Description \\
	
	\rowcolor{lightcyan} $\backslash$n & Newline. Position the screen cursor to the beginning of the next line.\\ 
	
	\rowcolor{lightcyan} $\backslash$t & Horizontal tab. Move the screen cursor to the next tab stop.\\ 
	
	\rowcolor{lightcyan} $\backslash$r & Carriage return. Position the screen cursor to the beginning of the current line; do not advance to the next line.\\ 
	
	\rowcolor{lightcyan} $\backslash$a & Alert. Sound the system bell.\\ 
	
	\rowcolor{lightcyan} $\backslash\backslash$ & Backslash. Used to print a backslash character.\\ 
	
	\rowcolor{lightcyan} $\backslash$' & Single quote. Used to print a single-quote character.\\ 
	
	\rowcolor{lightcyan} $\backslash$" & Double quote. Used to print a double-quote character.
\end{tabular}
\end{frame}


\begin{frame}{\small Arithmetic Operators}
	\centering\tiny\renewcommand{\arraystretch}{2}
	\begin{tabular}{p{0.15\linewidth} p{0.16\linewidth} p{0.2\linewidth} p{0.2\linewidth}}
		
		\rowcolor{cyan}\color{white} Operation & \color{white} Arithmetic operator & \color{white} Algebraic expression & \color{white} C++ expression \\
		
		\rowcolor{lightcyan} Addition & + & $f + 7$ & \texttt{f + 7} \\
		
		\rowcolor{lightcyan} Subtraction & - & $p - c$ & \texttt{p - c} \\
		
		\rowcolor{lightcyan} Multiplication & * & $bm$ or $b \cdot m$ & \texttt{b * m} \\
		
		\rowcolor{lightcyan} Division & / & $\dfrac{x}{y}$ or $x/y$ or $x\div y$ & \texttt{x / y} \\
		
		\rowcolor{lightcyan} Remainder & \% & $r$ $mod$ $s$ & \texttt{r \% s} 
	\end{tabular}
\end{frame}


\begin{frame}{\small Precedence of Arithmetic Operators}
	\centering\tiny\renewcommand{\arraystretch}{2}
	\begin{tabular}{p{0.12\linewidth} p{0.13\linewidth} p{0.65\linewidth}}
		
		\rowcolor{cyan}\color{white} Operator(s) & \color{white} Operation(s) & \color{white} Order of evaluation (precedence) \\\hline
		
		\rowcolor{lightcyan} \texttt{( )} & Parentheses & Evaluated first. For nested parentheses, such as in the expression
		\texttt{a * (b + c / (d + e))}, the expression in the innermost pair evalu ates first.\newline[Caution: If you have an expression such as \texttt{(a + b) * (c - d)} in which two sets of parentheses are not nested, but appear “on the same level,” the C++ Standard does not specify the order in which these parenthesized subexpressions will evaluate.] \\\hline
		
		\rowcolor{lightcyan} \texttt{*}\newline\texttt{/} \newline \texttt{\%} & Multiplication \newline Division \newline Remainder & Evaluated second. If there are several, they’re evaluated left to right. \\\hline
		
		\rowcolor{lightcyan} \texttt{+}\newline\texttt{-} & Addition \newline Subtraction & Evaluated last. If there are several, they’re evaluated left to right. \\\hline	
	\end{tabular}
\end{frame}



\begin{frame}{\small Decision Making: Equality and Relational Operators}
	\centering\tiny\renewcommand{\arraystretch}{2}
	\begin{tabular}{p{0.15\linewidth} p{0.15\linewidth} p{0.2\linewidth} p{0.25\linewidth}}
		
		\rowcolor{cyan}\color{white} Algebraic relational or equality operator & \color{white} C++ relational or equality operator & \color{white} Sample C++ condition & \color{white} Meaning of C++ condition \\\hline
		
		\rowcolor{lightcyan} \multicolumn{4}{l}{Relational operators} \\
		
		\rowcolor{lightcyan} $>$ & \texttt{>} & \texttt{x > y} & x is greater than y \\
		
		\rowcolor{lightcyan} $<$ & \texttt{<} & \texttt{x < y} & x is less than y \\
		
		\rowcolor{lightcyan} $\geq$ & \texttt{>=} & \texttt{x >= y} or & x is greater than or equal to y \\
		
		\rowcolor{lightcyan} $\leq$ & \texttt{<=} & \texttt{x <= y} & x is less than or equal to y \\\hline
		
		\rowcolor{lightcyan} \multicolumn{4}{l}{Equality operators} \\
		
		\rowcolor{lightcyan} $=$ & \texttt{==} & \texttt{x == y} & x is equal to y \\
		
		\rowcolor{lightcyan} $\neq$ & \texttt{!=} & \texttt{x != y} & x is not equal to y \\\hline
	\end{tabular}
\end{frame}


%
\end{document} 